% Please do not change the document class
\documentclass{scrartcl}

% Please do not change these packages
\usepackage[hidelinks]{hyperref}
\usepackage[none]{hyphenat}
\usepackage{setspace}
\doublespace

% You may add additional packages here
\usepackage{amsmath}

% Please include a clear, concise, and descriptive title
\title{The impacts of Agile Methodologies on the motivation within a team across a projects.}

% Please do not change the subtitle
\subtitle{COMP150 - Agile Development Practice}

% Please put your student number in the author field
\author{1507729}

\begin{document}

\maketitle

\abstract The Agile development methodology is the most prominent within the software development. It is at it's core, Agile is designed to be adaptable for any and every team following it's guidelines. However that can also lead teams failing to fully utilise the Agile philosophy. This paper will explore how the Agile method can impact the motivation of teams and the impact it has had on some existing teams.

\section{Introduction}

This paper will outline the motivational impacts of the Agile philosophy \cite{Agile} within a team by using case studies and other research that has been conducted around this topic. This paper will be excluding motivational factors like career prospects and job security where possible, though this are also areas to consider for any team they should be excluded for the sake of maintaining the integrity of the paper. 

For clarity, Motivation as described in the Oxford dictionary 1.1, is the ``Desire or willingness to do something; enthusiasm:''. It is this enthusiasm, the desire to work, that will be addressed. The definition of team that will be used is the one suggested by Hackman \cite{GroupExec}, but further expanded. To simplify it, it  states that ``A team is a collection of individuals who are interdependent in their tasks, who share responsibility for outcomes'' 


\section{Looking at Agile within Teams}
One of the central idea's suggested by Cockburn and Highsmith \cite{AgilePeople} is that Agile helps to improve motivation and morale. It does this by implementing increased communication and reduced documentation, this helps to build a sense of community within the team. However this could also lead to conflict as more interaction can also mean more chance for people to have arguments or disagreements. 

The work of Asproni \cite{MotivationSoftware} who explores the key motivational factors for software developers. Drawing the conclusion that a developer is more interested in the work being interesting, than being paid more. It's not difficult to assume that people working within the games industry being interested in their work is a huge benefit and will help to improve their enthusiasm for the project. 

McHugh, Conboy and Lang \cite {AgilePractices} show some interesting results in which already established teams , allowing them skip over as Tuckman \cite {Tuckman} suggests in his work the initial stages of Forming, Norming, etc. Reducing the possible effects on the teams motivations that can occur. In their research they found that many of the people who had experience from a more traditional management style, had little desire to work on a team that did not use an Agile approach.
 
The research from various authors shows that the key to productive teams is: to make sure you build a sense of community, to make sure that the team are involved in work they find interesting and to let them experience other types of management so they can decide which is best. However a poor implementation of any management style can result in a dislike and it's possible that although Agile was received well initially an study on to the effects of prolonged working with the Agile practices could then change the results.

\section{Agile in the Gaming Industry}

The adaptive nature of Agile means that it's well suited for software development. Some of the reflective work from software development industry its self \cite{Practitioners}, leads credence to the above research. However when looking within the gaming industry not all aspect of Agile can be considered the best option \cite{RyseCrisis} a reflective look at Agile development mainly on Scrum revealed that Agile was very useful at the initial stages of development, but as time progressed the need for a more structured or linear approach was necessary to ensure that deadlines were met. Though later expands on this point to say that being able to change to a different style of development could also be considered Agile in it's own way. 

There are many useful studies which have taken place for both the software and management side of using Agile, however the gaming industry combines several disciplines which need to co-operate. There is little that can been found in the way of quantified research upon the topic of motivation and morale within a game development company and the effects Agile has had upon them. 

Despite the lack of research into the subject matter it would seem that Agile improves, motivation within teams. I would infer that the it is a benefit to the team to introduce Agile practices during a developmental phase and that similar practices aren't potentially as suitable during a bug fixing or patching phase within a company though the daily stand-ups would of course keep the work flow from degrading.


\section{Conclusion}

In conclusion from the above information, Agile as a management system can be effective in software and games development. It's also important to note that there has been little research on management styles and their effects on motivation within the games industry exclusively. Though it is a topic that has been quite deeply explored in other works. \cite {EnhancePerformance, SoftwareProcess, LargeAgile, MicrosoftExp, UsingAgile, AgileDevelopment, MotivationTeamwork} Many of these works have valuable research in terms of raw quantifiable data into the subject of Agile within management however they also focus on other factors to motivation as proposed by Asproni.

Overall the it would appear that Agile is the ultimate philosophy, however as proposed by Payne not all situations are suitable for the Agile philosophy. He proposed that assessing your situation is an important part of making a successful project, and that Agile is about being adaptive. The example he gave was ``I wouldn't use a hammer to cut a piece of paper.''  The research I have done has often spoken about Agile as the only method and the greatest thing, it's important however to also recognize that every group is different and people need to have different team setting in order to reach their full potential.

\bibliographystyle{ieeetran}
\bibliography{references}

\end{document}
