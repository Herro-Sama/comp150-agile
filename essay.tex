% Please do not change the document class
\documentclass{scrartcl}

% Please do not change these packages
\usepackage[hidelinks]{hyperref}
\usepackage[none]{hyphenat}
\usepackage{setspace}
\doublespace

% You may add additional packages here
\usepackage{amsmath}

% Please include a clear, concise, and descriptive title
\title{The impact of the Agile Methodology on Team Morale and how to manage it.}

% Please do not change the subtitle
\subtitle{COMP150 - Agile Development Practice}

% Please put your student number in the author field
\author{Student Number - 1507729}

\begin{document}

\maketitle

\abstract The agile development methodology is the most prominent within the software development. It is at it's core designed to be agile and to be adaptable to each team. However that can also lead teams failing to fully utilise the agile methodology  in this paper I would like to explore how the agile method can be best used to maximise the morale of a team and to maintain that morale across a project. (72 Words)

\section{Introduction}

The Agile development methodology \cite{Agile} is the most commonly used software development practice, and paired along with the Scrum methodology it has been shown in several case studies to have improved efficiency of teams in the software development sector. However does this improved efficiency have an impact on the team morale? In this paper I will explore how agile and scrum will impact the overall morale of the team, with the use of existing case studies and from various other researchers and the insights from their experiments. 

Ignoring the various flawed implementations of agile, an example being what Martin Fowler describes as a "Flaccid Scrum".Then focusing only on the variation of the agile methodology tailored to best suit the team, it's in this optimal state that I wish to explore a 'working' build of agile teams that have reached their "performing" stage to quote Bruce Tuckman. 

\section{An Example Case study}

With motivation and morale being the subject matter of this paper I am going to infer from an existing case study. "Cite: Using Agile Practices to Solve Global Software Development Problems -- A Case Study" The mentioned paper looks at some of the ways agile can help resolve Global software problems that arise, in one section they talk about the impact such work has on the motivation. Not all of the issues that were raised in this paper were relevant however, an example being staff working offsite not being able to attend additional training days, and the chance of advancement within the business. One of the main issues raised was staff with various talents are more inclined to want to use their talents and with self-organising teams, these staff members are able to choose which work will be best suited to their individual talents preventing such issues.

There is little else talked about through this study and no information given in the conclusion about the morale implications agile has had on the team. It's easy to infer that some aspects of agile have greatly improved the efficiency of the business, but little information is given as to the level of satisfaction for staff members.

Write the main body of your essay here. Add more sections if appropriate. You may choose to write about each of your three papers in its own section, or you may choose a different structure. Either way, remember that you are being assessed on technical insight and analysis: it is not enough to merely summarise the contents of the three papers. You must demonstrate the ability to make inferences beyond what is written in the papers, and to draw the three papers together into a single coherent narrative.

Your essay must make a clear recommendation, in terms of which of the three techniques you have reviewed is the best according to whichever metric or metrics you feel is most appropriate. You must justify your choice, backing it up with empirical evidence. However remember that an academic essay is not a murder mystery: you should already have briefly discussed your recommendation in the introduction and in other parts of the essay. Do not save it for a grand reveal at the end.

\section{Iterative Works}


\section{Conclusion}

Write your conclusion here. The conclusion should do more than summarise the essay, making clear the contribution of the work and highlighting key points, limitations, and outstanding questions. It should not introduce any new content or information.

\bibliographystyle{ieeetran}
\bibliography{references}

\end{document}
